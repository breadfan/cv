\documentclass[11pt,a4paper]{moderncv}

\moderncvtheme[green]{casual}
\usepackage[utf8]{inputenc}
\usepackage[scale=0.8]{geometry}

\usepackage[unicode]{hyperref}
\definecolor{linkcolour}{rgb}{0,0.2,0.6}
\hypersetup{colorlinks,breaklinks,urlcolor=linkcolour, linkcolor=linkcolour}

\firstname{Danil}
\familyname{Kizeev}
\address{}{Saint Petersburg, Russia}
\mobile{+79040283459}
\email{kizeevdanil@yandex.ru}
%\extrainfo{Software should be beautiful. Both inside and outside.}

\makeatletter
\renewcommand*{\bibliographyitemlabel}{\@biblabel{\arabic{enumiv}}}
\makeatother

\begin{document}
\maketitle

\section{Keywords}
\cvline
{}{python, numpy, pandas, SQL, PostgreSQL, sklearn, scipy, transformers, matplotlib, seaborn, pytorch, C++, optimization, Machine Learning, Data Science, Classification, Regression, Automatic Categorization, NLP.}
\section{Projects}
  \subsection{Study projects/pet projects}
  \cvline
	  {Diploma}
	  {\url{https://github.com/breadfan/Bachelor-Thesis}\newline{}
	  	\textbf{Applying BERT for automatic posts categorization in social network.} \newline{}
	  	Using BERT, BERTopic and word2vec + TF-IDF for categorization. \newline{}
	  	Upgrading quality of models from BERT to word2vec.\newline{}
	  	With given categories need to make mapping from amount of posts to that categories for simple recommendations.
	  }
  \cvline
	  {Sarcasm detection}
	  {\url{https://www.kaggle.com/taciturno/sarcasm-detection-with-logit}\newline{}
	  \textbf{Sarcasm detection via \url{reddit.com} dataset of users}\newline{}
	  Using commentaries of users need to classify if comment sarcastic or not. Work mainly produced using \textbf{TF-IDF + logistic regression} pipeline. For EDA \textbf{plotly} library was used mainly.}
  \cvline
    {Accelerated MDM-method}
    {\url{https://github.com/breadfan/Accelerated_MDM_method}\newline{}
    \textbf{Researching Acceleration of an MDM-method}\newline{}
    As a course work for third year two methods were implemented: MDM and accelerated MDM methods with visualization for 2- and 3-dim. cases for running time comparison.}
  \cvline
    {Polynomial separation}
    {\url{https://github.com/breadfan/polynomial_separation}\newline{}
    \textbf{Project}\newline{}
    Using linear programming for polynomial separation problem with given two sets of dots.
    }
  \cvline
    {Kaggle}
    {\url{https://www.kaggle.com/taciturno/account}\newline{}
    \textbf{Some Kaggle notebooks via mlcourse.ai homework's}\newline{}
    }
  	
  
\pagebreak
\section{Experience}
	---

\section{Technical Skills}
\cvline
  {Languages}{C$\backslash$C$++$, Python, R, Java, C\#}
\cvline
  {VCS}{Git}
\cvline
  {OS}{Windows, Linux (Debian)}
%\cvline
  %{Techniques}{TDD, XP, Scrum, Kanban}

\section{Education}
  \subsection{University}
  \cventry
	  {Sep 2021 - Jun 2023}
	  {Data Analytics, Master}
	  {Saint-Petersburg National Research University of Information Technologies, Mechanics and Optics, Russia}
	  {}{}{}
  \cventry
	  {Sep 2018 - Jun 2021}
	  {Applied mathematics and computer science, Bachelor}
	  {Saint Petersburg State University, Russia}
	  {}{}{}
  \cventry
	  {Sep 2017 - Sep 2018}
	  {Cybersecurity department, Bachelor}
	  {Tver State University, Russia}
	  {}{}{}
  
  \subsection{MOOCs (stepik.org, mlcourse.ai, intuit.ru)}
  \cventry
	  {Feb 2021 - Feb 2021}
	  {Introduction in geometric programming}
	  {\newline Learned basics in optimization problems, which goal functions is posynomial --- Geometic Programming (GP) problems.}
	  {}{\newline\url{https://intuit.ru/verifydiplomas/101428913}}{}
  \cventry
	  {Jun 2020 - Jun 2021}
	  {Machine Learning Course}
	  {\newline Learning current time machine learning basics}
	  {}{\newline\url{https://mlcourse.ai}}{}
  \cventry
	  {Sep 2019 - Jan 2020}
	  {Elements of financial mathematics}
	  {\newline Learned fundamentals of mathematics in finances, calculation methods for making desicions on financial field}
	  {}{\newline\url{https://intuit.ru/verifydiplomas/101301190}}{}
  \cventry
	  {Sep 2018 - Nov 2018}
	  {Basics of programming with R}
	  {\newline Learned basics of R and vectorization features through some hard tasks}
	  {}{\newline\url{https://stepik.org/course/497}}{}
  \cventry
    {May 2018 - Jun 2018}
    {Programming on Python}
    {\newline Learned basics of Python}
    {}{\newline\url{https://stepik.org/course/67}}{}
  \cventry
    {Apr 2015 - Jun 2015}
    {Algorithms: theory and practice. Methods.}
    {\newline Learned basics of algorithms and practiced it}
    {}{\newline\url{https://stepik.org/course/217}}{}
  \cventry
    {Jun 2017 - Aug 2017}
    {Introducion to Programming}
    {\newline Learned basics of programming and C++ features}
    {}{\newline\url{https://stepik.org/course/363/}}{}


  \subsection{On the Internet}
    
    \cvline{GitHub}{\url{https://github.com/breadfan}}
    \cvline{StackOverflow}{\url{https://stackoverflow.com/users/9850300/taciturno}}
    \cvline{LinkedIn}{\url{https://www.linkedin.com/in/danil-kizeev-57a48bba}}

\end{document}
